%% LaTeX-Beamer template for KIT design
%% by Erik Burger, Christian Hammer
%% title picture by Klaus Krogmann
%%
%% version 2.1
%%
%% mostly compatible to KIT corporate design v2.0
%% http://intranet.kit.edu/gestaltungsrichtlinien.php
%%
%% Problems, bugs and comments to
%% burger@kit.edu

\documentclass[18pt]{beamer}

\usepackage[utf8]{inputenc}
\usepackage[babel,german=quotes]{csquotes}
\usepackage{graphicx}
\usepackage{caption}
\usepackage{subfig}
\usepackage[right]{eurosym}
\usepackage{listings}

%% SLIDE FORMAT

% use 'beamerthemekit' for standard 4:3 ratio
% for widescreen slides (16:9), use 'beamerthemekitwide'

\usepackage{templates/beamerthemekit}
% \usepackage{templates/beamerthemekitwide}

%% TITLE PICTURE

% if a custom picture is to be used on the title page, copy it into the 'logos'
% directory, in the line below, replace 'mypicture' with the 
% filename (without extension) and uncomment the following line
% (picture proportions: 63 : 20 for standard, 169 : 40 for wide
% *.eps format if you use latex+dvips+ps2pdf, 
% *.jpg/*.png/*.pdf if you use pdflatex)

\titleimage{title}

%% TITLE LOGO

% for a custom logo on the front page, copy your file into the 'logos'
% directory, insert the filename in the line below and uncomment it

\titlelogo{titlelogo}

% (*.eps format if you use latex+dvips+ps2pdf,
% *.jpg/*.png/*.pdf if you use pdflatex)

%% TikZ INTEGRATION

% use these packages for PCM symbols and UML classes
% \usepackage{templates/tikzkit}
% \usepackage{templates/tikzuml}

% the presentation starts here

\title[C++ Workshop]{C++ Workshop}
\subtitle{2. Block, 04.05.2012}
\author{Robert Schneider, Sven Brauch}

\institute{}

\begin{document}

% change the following line to "ngerman" for German style date and logos
\selectlanguage{ngerman}

\AtBeginSection[]{%
	\begin{frame}
		\tableofcontents[sectionstyle=show/hide,subsectionstyle=hide/show/hide]
	\end{frame}
	\addtocounter{framenumber}{-1}% If you don't want them to affect the slide number
}

%title page
\begin{frame}
\titlepage
\end{frame}

%table of contents
\begin{frame}{Gliederung}
\tableofcontents
\end{frame}

\section{Objektorientierte Programmierung}
\begin{frame}
    \frametitle{Structures}
    \begin{block}{structs}
    Ein \texttt{struct} ist eine Zusammenfassung mehrerer Objekte zu einem größeren.
    Zum Beispiel könnte man ein \texttt{struct} "`Quader"' erstellen, welches drei Fließkommazahlen beinhaltet.
    \end{block}
    \begin{block}{Instanzen}
    Eine solche \texttt{struct} ist lediglich eine abstrakte Beschreibung des Objekts; man arbeitet schließlich mit sogenannten \emph{Instanzen} des Objekts. Beispiel Quader: Die Structure an sich beschreibt das abstrakte Objekt, die Instanz einen konkreten Quader ("`der Quader auf meinem Tisch"').
    \end{block}
\end{frame}
\begin{frame}
    \frametitle{Beispiel für eine Structure}
    \vspace{0.7cm}
    \includegraphics[width=15cm]{example_code/box.pdf}
\end{frame}

\begin{frame}
    \frametitle{Klassen}
    \begin{block}{Klassen}
    Eine \texttt{class} ist eine \texttt{struct}, die zusätzlich zu Daten noch Funktionen enthält, die auf diesen Daten operieren.
    \end{block}
    \begin{block}{Konstruktor und Destruktor}
    Eine Klasse hat zwei besondere Funktionen, den Konstruktor und den Destruktor; der Konstruktor wird aufgerufen, wenn eine neue Instanz der Klasse erstellt wird, und der Destruktor, wenn die Instanz wieder gelöscht wird. Der Konstruktor heißt \texttt{klassenname}, der Destruktor \texttt{~klassenname}.
    \end{block}
\end{frame}
\begin{frame}
    \frametitle{Beispiel für eine Klasse}
    \vspace{0.7cm}
    \includegraphics[width=10cm]{example_code/box2.pdf}
\end{frame}
\begin{frame}
    \frametitle{Etwas anderes Beispiel für eine Klasse}
    \vspace{0.7cm}
    \includegraphics[width=10cm]{example_code/box3.pdf}
\end{frame}

%%%%%%%%%%%%%%%%%%%%%%%%%
% ADD OWN SECTIONS HERE %
%%%%%%%%%%%%%%%%%%%%%%%%%
%\section{C++}


\begin{frame}{Was ist C++?}
	\begin{block}{Eine standardisierte Programmiersprache}
		C++
		\begin{itemize}
			\item ist eine Programmiersprache (gibt einen Programmablauf vor)
			\item vereint sowohl high-level- wir auch low-level-Features
			\item ist sehr hardwarenah
			\item ist standardisiert (\emph{hier:} ISO/IEC 14882:2003)
		\end{itemize}
	\end{block}
	
	\pause
	
	\begin{block}{Was sagt der Standard hierzu?}
		C++ is a general purpose programming language based on the C programming language as described in
		ISO/IEC 9899:1990 Programming languages – C.
	\end{block}
\end{frame}

\begin{frame}{Worum geht es bei C++?}
	von-Neumann-Architektur $\xrightarrow{Abstraktion}$ C++ $\xrightarrow{Implementierung}$ Hardware
	
	\begin{itemize}
		\item C++ wurde explizit für die von-Neumann-Architektur designed
		\item C++ soll zero-overhead sein (Features, die ich nicht nutze, benötigen weder Speicher noch Laufzeit)
		\item C++ ist eine Multi-Paradigmen-Sprache mit Betonung der Objektorientierung
	\end{itemize}
	
	\begin{block}{Standard, 1.8}
		The constructs in a C++ program create, destroy, refer to, access, and manipulate objects.
	\end{block}
\end{frame}

\begin{frame}{Wofür nutze ich C++?}
	foo!
\end{frame}

\begin{frame}[fragile]{»Dinge«}
	Das grundlegende Konzept in C++ nennt sich im Standard \enquote{object}. Hinsichtlich Java und Objektorientierung aber missverständlich!
	Wir nennen es daher »Ding«.
	
	\pause
	
	\small
	\begin{block}{Standard, 1.8}
		Ein »Ding«
		\begin{itemize}
			\item ist ein Speicherbereich, aber \emph{keine} Funktion (auch wenn diese Speicher belegt!).
			\item wird durch eine Definition, den \verb|new|-Ausdruck oder vom Compiler erzeugt.
			\item hat einen Typen und eine Speicherdauer {\tiny (die die Lebensdauer des dort gespeicherten »Objekts« beeinflusst)}; \emph{kann} einen \emph{Namen} haben.
			\item hat eine Größe von einem oder mehr Bytes {\tiny (abgesehen von bit-fields)}.
			\item von »einfachem« {\tiny (POD)} Typ besetzt eine zusammenhängende Menge Bytes.
		\end{itemize}
	\end{block}
\end{frame}

\begin{frame}[fragile]{Speicher}
	\begin{block}{Standard, 1.7}
		Die fundamentale Speicher-Einheit im C++ Speichermodell ist das \emph{Byte}. [Es folgt eine sehr abstrakte Definition.]
		Der Speicher, welcher einem C++ Programm zur Verfügung steht, besteht aus einer oder mehreren Sequenzen von zusammenhängenden Bytes.
		Jedes Byte hat eine eindeutige Adresse.
	\end{block}
	
	\footnotesize
	\begin{block}{}
		\begin{lstlisting}[language=C++]
			int foo;
			int bar;
			double d;
			double& rd = d;
			
			cout << sizeof(foo);
			cout << sizeof(int);
		\end{lstlisting}
	\end{block}
\end{frame}

%, memory model, object model, Ring HW-Standard-Impl % includes cpp.tex

\end{document}
